%%%%%%%%%%%%%%%%%%%%%%%%%%%%%%%%%%%%%%%%%%%%%%%%%%%%%%%%%%%%%%
% MACROS
% 2017/10/13 Sex 11:34:37 
%%%%%%%%%%%%%%%%%%%%%%%%%%%%%%%%%%%%%%%%%%%%%%%%%%%%%%%%%%%%%%%%%

%%%%%%%%%%%%%%%%%%%%%%%%%%%%%%%%%%%%%%%%%%%%%%%%%%%%%%
% pagestyle
\rhead{\thepage}
\lhead{\theauthor}
\chead{\thetitle}
\renewcommand{\headrulewidth}{.5 pt} % no rule 
\rfoot{}
\lfoot{}
\cfoot{}

%%%%%%%%%%%%%%%%%%%%%%%%%%%%%%%%%%%%%%%%%%%%%%%%%%%%%%%%%%%%%%%%%
% Maketitle customization
\renewcommand{\maketitle}{%
\begin{center}
{\Large\uppercase{\thetitle}}\\
\vspace{1 em}
{\large\textsc{By \theauthor}} \\
\vspace{1 em}
\thedate
\end{center}
}

%%%%%%%%%%%%%%%%%%%%%%%%%%%%%%%%%%%%%%%%%%%%%%%%%%%%%%%%%%%%%%%%%
% MATH MACROS

%%%%%%%%%%%%%%%%%%%%%%%%%%%%%%%%%%%%%%%%%%%%%%%%%%%%%%%%%%%%%%%%%
% inner product
% produto interno na forma <x,y>
% \DeclarePairedDelimiterX\innerp[2]{\langle}{\rangle}{#1,#2}

%%%%%%%%%%%%%%%%%%%%%%%%%%%%%%%%%%%%%%%%%%%%%%%%%%%%%%%%%%%%%%%%%
% sequencias
%  fazer $\{x_n\}$ e $\{x\}_n^N$ mais rápido.
\renewcommand{\seq}[1]{\{#1\}}

%%%%%%%%%%%%%%%%%%%%%%%%%%%%%%%%%%%%%%%%%%%%%%%%%%%%%%%%%%%%%%%%%
% Atalhos para notação de conjuntos dos números reais, naturais, etc
% \renewcommand{\R}{\mathbb{R}}
\newcommand{\model}{\ensuremath{\mathcal{M}}}
\newcommand{\RS}{\ensuremath{R^2}}
\newcommand{\N}{\ensuremath{\mathbb{N}}}
\newcommand{\Q}{\ensuremath{\mathbb{Q}}}
\newcommand{\R}{\ensuremath{\mathbb{R}}} 
\newcommand{\Rn}{\ensuremath{\mathbb{R}^n}} 
\newcommand{\Rm}{\ensuremath{\mathbb{R}^m}} 
\newcommand{\Rp}{\ensuremath{\mathbb{R}^p}} 
\newcommand{\I}{\ensuremath{\mathbb{R}\setminus\mathbb{Q}}}
\newcommand{\md}{\ensuremath{\mathcal{D}}}

%%%%%%%%%%%%%%%%%%%%%%%%%%%%%%%%%%%%%%%%%%%%%%%%%%%%%%%%%%%%%%%
% atalhos para outros operadores matemáticos
\DeclareMathOperator{\lagr}{\mathcal{L}}
\DeclareMathOperator{\eu}{\mathrm{e}}
\DeclareMathOperator{\de}{\mathrm{d}}
\DeclareMathOperator{\up}{\uparrow}
\DeclareMathOperator{\dn}{\downarrow}

%%%%%%%%%%%%%%%%%%%%%%%%%%%%%%%%%%%%%%%%%%%%%%%%%%%%%%%%%%%%%%%
% Derivadas
\newcommand{\dpf}[2]{\frac{d#1}{d#2}}
\newcommand{\dsf}[2]{\frac{d^2#1}{d#2^2}}
\newcommand{\opt}[1]{{#1}^{*}}

%%%%%%%%%%%%%%%%%%%%%%%%%%%%%%%%%%%%%%%%%%%%%%%%%%%%%%%%%%%%%%%
% Derivadas Parciais
\newcommand{\pdf}[2]{\frac{\partial#1}{\partial#2}}
\newcommand{\psdf}[2]{\frac{\partial^2#1}{\partial#2^2}}
\newcommand{\pcdf}[3]{\frac{\partial^2#1}{\partial#2 \partial#3}}

%%%%%%%%%%%%%%%%%%%%%%%%%%%%%%%%%%%%%%%%%%%%%%%%%%%%%%%%%%%%%%%
% função valor
\newcommand{\vth}{V_{\theta}}

%%%%%%%%%%%%%%%%%%%%%%%%%%%%%%%%%%%%%%%%%%%%%%%%%%%%%%%%%%%%%%%
% Atalhos para cores:

\newcommand{\red}[1]{\textcolor{red}{#1}}
\newcommand{\blue}[1]{\textcolor{blue}{#1}}
\newcommand{\green}[1]{\textcolor{green}{#1}}
\newcommand{\magenta}[1]{\textcolor{magenta}{#1}}
% \newcommand{\hl}[1]{\colorbox{yellow}{#1}}

%%%%%%%%%%%%%%%%%%%%%%%%%%%%%%%%%%%%%%%%%%%%%%%%%%%%%%%%%%%%%%%
% GREEK LETTERS
\newcommand{\ga}{\ensuremath{\alpha}} 
\newcommand{\gab}{\ensuremath{\boldsymbol{\alpha}}} 
\newcommand{\gb}{\ensuremath{\beta}} 
\newcommand{\gbb}{\ensuremath{\boldsymbol{\beta}}} 
\newcommand{\G}{\ensuremath{\Gamma}} 
\newcommand{\gg}{\ensuremath{\gamma}} 
\newcommand{\D}{\ensuremath{\Delta}} 
\newcommand{\gl}{\ensuremath{\lambda}} 
\newcommand{\gL}{\ensuremath{\Lambda}} 
\newcommand{\gSi}{\ensuremath{\Sigma^{-1}}}  % inv sigma
\newcommand{\gS}{\ensuremath{\Sigma}} 
\newcommand{\gs}{\ensuremath{\sigma}} 
\newcommand{\gsh}{\ensuremath{\hat{\sigma}}} 
\newcommand{\gss}{\ensuremath{\sigma^2}} 
\newcommand{\gssh}{\ensuremath{\hat{\sigma}^2}}  % sigma hat
\newcommand{\gSib}{\ensuremath{\boldsymbol{\Sigma}^{-1}}} 
\newcommand{\gSb}{\ensuremath{\boldsymbol{\Sigma}}} 
\newcommand{\gsb}{\ensuremath{\boldsymbol{\sigma}}} 
\newcommand{\gib}{\ensuremath{\boldsymbol{\iota}}} 
\newcommand{\gi}{\ensuremath{\iota}}
\newcommand{\err}{\ensuremath{\varepsilon}} 
\newcommand{\gO}{\ensuremath{\Omega}} 

%%%%%%%%%%%%%%%%%%%%%%%%%%%%%%%%%%%%%%%%%%%%%%%%%%%%%%%%%%%%%%%
% 
\newcommand{\bm}[1]{\boldsymbol{#1}}
\newcommand{\mbs}[1]{\boldsymbol{#1}}
\newcommand{\mbf}[1]{\mathbf{#1}}


%%%%%%%%%%%%%%%%%%%%%%%%%%%%%%%%%%%%%%%%%%%%%%%%%%%%%%%%%%%%%%%
% Não faço a mínima ideia o que esse comandos fazem
\renewcommand{\cancel}{\color{red}}

\newcommand{\encircle}[1]{\tikz[baseline= (X.base)]\node(X)[draw, shape=circle, inner sep=0]{\strut#1};}
%%%%%%%%%%%%%%%%%%%%%%%%%%%%%%%%%%%%%%%%%%%%%%%%%%%%%%%%%%%%%%%



