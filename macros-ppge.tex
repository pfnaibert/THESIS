%%%%%%%%%%%%%%%%%%%%%%%%%%%%%%%%%%%%%%%%%%%%%%%%%%%%%%
% Sumario (TOC) ABNT-6027-2012
% CFT == Table of Contents, Figures and Tables 

% Comando para mudar letras dos títulos nas CFT

% HEADSTART
\renewcommand{\tocheadstart}{\vspace{0 em}\center}
\renewcommand{\lofheadstart}{\vspace{0 em}\center}
\renewcommand{\lotheadstart}{\vspace{0 em}\center}

% AFTER TITLE
\renewcommand{\afterloftitle}{\par\vspace{\baselineskip}}
\renewcommand{\afterlottitle}{\par\vspace{\baselineskip}}
\renewcommand{\aftertoctitle}{\par\vspace{-0.5 em}}

% lista de ilustracoes
\renewcommand{\cftfigurename}{\figurename\space} 
\renewcommand*{\cftfigureaftersnum}{\hfill\textendash\hfill}
 
% lista de tabelas  
\renewcommand{\cfttablename}{\tablename\space}
\renewcommand*{\cfttableaftersnum}{\hfill\textendash\hfill}	

% suprime espaço entre os itens de listings de diferentes capítulos 
\setlength{\cftbeforechapterskip}{0pt plus 0pt}
\renewcommand*{\insertchapterspace}{}

% pagestyle (layout de página)
\addtodef{\listoffigures}{\clearpage\pagestyle{empty}}{}
\addtodef{\listoftables}{\clearpage\pagestyle{empty}}{}
\addtodef{\tableofcontents}{\clearpage\pagestyle{empty}}{}

% nivel de profundidade (implementa o quinto nivel - subsubsubsection)
\setcounter{secnumdepth}{4}
\setcounter{tocdepth}{2}

% leaders para os capitulos (pontos entre titulo e pagenum)
\renewcommand*{\cftchapterdotsep}{\cftdotsep} 

% macros para configurar o Sumário conforme 6027-2012 ou tradicional cft
\cftinsertcode{A}{}

% Set indents
\newlength{\cftlastnumwidth}
\setlength{\cftlastnumwidth}{\cftsubsectionnumwidth}
\addtolength{\cftlastnumwidth}{0 em}

\cftsetindents{part}{0em}{\cftlastnumwidth}
\cftsetindents{chapter}{0em}{\cftlastnumwidth}
\cftsetindents{section}{0em}{\cftlastnumwidth}
\cftsetindents{subsection}{0em}{\cftlastnumwidth}
\cftsetindents{subsubsection}{0em}{\cftlastnumwidth}
\cftsetindents{paragraph}{0em}{\cftlastnumwidth}
\cftsetindents{subparagraph}{0em}{\cftlastnumwidth}
\cftsetindents{bibsection}{0em}{\cftlastnumwidth}

\renewcommand{\tocprintchapter}{%
\addtocontents{toc}{\cftsetindents{chapter}{0em}{\cftlastnumwidth}}}

\renewcommand{\tocinnonumchapter}{%
\addtocontents{toc}{\cftsetindents{chapter}{\cftlastnumwidth}{2em}}%
\cftinserthook{toc}{A}}

\renewcommand{\tocprintchapternonum}{%
\addtocontents{toc}{\cftsetindents{chapter}{\cftlastnumwidth}{2em}}}

\renewcommand{\tocpartapendices}{%
\addtocontents{toc}{\cftsetindents{part}{\cftlastnumwidth}{2em}}
\cftinserthook{toc}{A}}

\renewcommand{\tocpartanexos}{%
\addtocontents{toc}{\cftsetindents{part}{\cftlastnumwidth}{2em}}
\cftinserthook{toc}{A}}

% separação entre os titulos dos capitulos na TOC
% \setlength{\cftbeforechapterskip}{2 em}
\setlength{\cftbeforechapterskip}{1 em}

% fonte das partes e do capitulo do TOC
\renewcommand{\cftchapterfont}{\normalsize\bfseries\MakeTextUppercase}
\renewcommand{\cftsectionfont}{\normalsize\MakeTextUppercase}
\renewcommand{\cftsubsectionfont}{\normalsize\bfseries}

%%%%%%%%%%%%%%%%%%%%%%%%%%%%%%%%%%%%%%%%%%%%%%%%%%%%%%
% Ajuste entrada Referencias

\renewcommand{\bibsection}{%
\chapter*{\bibname}
\bibmark
\phantomsection
\addcontentsline{toc}{chapter}{\texorpdfstring{\chapternumberline {}{\bibname}}}
% \addcontentsline{toc}{chapter}{\texorpdfstring{\chapternumberline {} {\bibname}}
\prebibhook
}

%%%%%%%%%%%%%%%%%%%%%%%%%%%%%%%%%%%%%%%%%%%%%%%%%%%%%%
% capitulos textual
% Alterado do AbneteX
\makechapterstyle{ppge1}{%
\renewcommand{\chapterheadstart}{\vspace{0 em}}
\newlength{\chapternamenumlength}

% tamanhos de fontes de chapter e part	
\setlength{\beforechapskip}{0pt}
\renewcommand{\chaptitlefont}{\normalfont\bfseries}
\renewcommand{\chapnumfont}{\chaptitlefont}

% impressao do numero do capitulo
\renewcommand{\chapternamenum}{}
\renewcommand{\printchapternum}{\chapnumfont\space\thechapter\space}
\renewcommand{\afterchapternum}{}

% impressao do nome do capitulo
\renewcommand{\printchaptername}{\chaptitlefont}

% impressao do titulo do capitulo
\def\printchaptertitle##1{%
\chaptitlefont%
\settowidth{\chapternamenumlength}{\printchaptername\printchapternum\afterchapternum}%
\parbox[t]{\columnwidth-\chapternamenumlength}{##1}}

\renewcommand{\afterchaptertitle}{\par \vspace{\baselineskip}}
}

%%%%%%%%%%%%%%%%%%%%%%%%%%%%%%%%%%%%%%%%%%%%%%%%%%%%%%
% capitulos pre and post textual
\makechapterstyle{ppge2}{%
\renewcommand{\chapterheadstart}{\vspace{0 em}}
\renewcommand{\chapnamefont}{\bfseries}
\renewcommand{\chapnumfont}{\bfseries}
\renewcommand{\chaptitlefont}{\centering\bfseries\MakeTextUppercase}

\renewcommand{\printchaptername}{}
\renewcommand{\chapternamenum}{}

\renewcommand{\printchapternum}{\chapnumfont\thechapter}
\renewcommand{\afterchapternum}{\hspace{1 em}}
\renewcommand{\afterchaptertitle}{\par \vspace{\baselineskip}}
}
\chapterstyle{ppge2}


%%%%%%%%%%%%%%%%%%%%%%%%%%%%%%%%%%%%%%%%%%%%%%%%%%%%%%
% Declaracoes de cabecalhos e Layout de Páginas

% Cabecalho do inicio do capitulo
\makepagestyle{abntchapfirst}
\makeoddhead{abntchapfirst}{}{}{\footnotesize{\thepage}}

% Cabecalho padrao
\makepagestyle{abntheadings}
\makeevenhead{abntheadings}{\thepage}{}{}
\makeoddhead{abntheadings}{}{}{\footnotesize{\thepage}}
% \makeheadrule{abntheadings}{\textwidth}{\normalrulethickness}

% layout para elementos pre-textuais
\newcommand{\pretextual}{%
  \aliaspagestyle{chapter}{empty}% customizing chapter pagestyle
  \pagestyle{empty}
  \aliaspagestyle{cleared}{empty}
  \aliaspagestyle{part}{empty}
  \chapterstyle{ppge2}
}
\renewcommand{\frontmatter}{\pretextual} 

% layout para elementos textuais
\newcommand{\textual}{%
%   \ifthenelse{\equal{\ABNTEXisarticle}{true}}{%
%     \pagestyle{plain}%
%   }{%else  
    \pagestyle{abntheadings}%
    \aliaspagestyle{chapter}{abntchapfirst}% customizing chapter pagestyle
     \chapterstyle{ppge1}
     % \chapterstyle{hangnum}
     % \chapterstyle{article}
     % \chapterstyle{section}
     % \chapterstyle{tandh}
% }  
  % \nouppercaseheads%
  % \bookmarksetup{startatroot}% 
}
\renewcommand{\mainmatter}{\textual}

% layout para elementos pos-textuais
\newcommand{\postextual}{%
\phantompart
\chapterstyle{ppge2}
}
\renewcommand{\backmatter}{\postextual}

%%%%%%%%%%%%%%%%%%%%%%%%%%%%%%%%%%%%%%%%%%%%%%%%%%%%%%
% Espaçamentos entre seções
% section
\setbeforesecskip{18 pt}
\setaftersecskip{18 pt}

% % subsection
\setbeforesubsecskip{18 pt}
\setaftersubsecskip{18 pt}

% subsubsection
\setbeforesubsubsecskip{\baselineskip}
\setaftersubsubsecskip{\baselineskip}

%%%%%%%%%%%%%%%%%%%%%%%%%%%%%%%%%%%%%%%%%%%%%%%%%%%%%%
% Fontes de section, subsec, subsubsec
\setsecheadstyle{\normalsize\MakeTextUppercase}
\setsubsecheadstyle{\normalsize\bfseries}
\setsubsubsecheadstyle{\normalsize\itshape}

%%%%%%%%%%%%%%%%%%%%%%%%%%%%%%%%%%%%%%%%%%%%%%%%%%%%%%
% O tamanho do parágrafo é dado por:
\setlength{\parindent}{1.3cm} % Tamanho do TAB

% Controle do espaçamento entre um parágrafo e outro:
% \setlength{\parskip}{0.2cm}  % tente também \onelineskip

%%%%%%%%%%%%%%%%%%%%%%%%%%%%%%%%%%%%%%%%%%%%%%%%%%%%%%
% Comando para mudar tamanho da paginação
% \renewcommand*{\thepage}{\footnotesize\arabic{page}}
% \renewcommand{\thepage}{\arabic{page}}

%%%%%%%%%%%%%%%%%%%%%%%%%%%%%%%%%%%%%%%%%%%%%%%%%%%%%%%%%%%%%%%
% Atalhos para cores:

\newcommand{\red}[1]{\textcolor{red}{#1}}
\newcommand{\blue}[1]{\textcolor{blue}{#1}}
\newcommand{\green}[1]{\textcolor{green}{#1}}
\newcommand{\magenta}[1]{\textcolor{magenta}{#1}}
% \newcommand{\hl}[1]{\colorbox{yellow}{#1}}

%%%%%%%%%%%%%%%%%%%%%%%%%%%%%%%%%%%%%%%%%%%%%%%%%%%%%%%%%%%%%%%%%
% MATH MACROS

%%%%%%%%%%%%%%%%%%%%%%%%%%%%%%%%%%%%%%%%%%%%%%%%%%%%%%%%%%%%%%%%%
% inner product
% produto interno na forma <x,y>
\DeclarePairedDelimiterX\innerp[2]{\langle}{\rangle}{#1,#2}

% floor function
\DeclarePairedDelimiterX\floor[1]{\lfloor}{\rfloor}{#1}

%%%%%%%%%%%%%%%%%%%%%%%%%%%%%%%%%%%%%%%%%%%%%%%%%%%%%%%%%%%%%%%%%
% sequences
%  fazer $\{x_n\}$ e $\{x\}_n^N$ mais rápido.
\renewcommand{\seq}[1]{\{#1\}}

%%%%%%%%%%%%%%%%%%%%%%%%%%%%%%%%%%%%%%%%%%%%%%%%%%%%%%%%%%%%%%%%%
% Atalhos para notação de conjuntos dos números reais, naturais, etc
% \renewcommand{\R}{\mathbb{R}}
\newcommand{\model}{\ensuremath{\mathcal{M}}}
\newcommand{\N}{\ensuremath{\mathbb{N}}}
\newcommand{\Q}{\ensuremath{\mathbb{Q}}}
\newcommand{\R}{\ensuremath{\mathbb{R}}} 
\newcommand{\I}{\ensuremath{\mathbb{R}\setminus\mathbb{Q}}}
\newcommand{\md}{\ensuremath{\mathcal{D}}}

%%%%%%%%%%%%%%%%%%%%%%%%%%%%%%%%%%%%%%%%%%%%%%%%%%%%%%%%%%%%%%%
% atalhos para outros operadores matemáticos
\DeclareMathOperator{\L}{\mathcal{L}}
\DeclareMathOperator{\eu}{\mathrm{e}}
\DeclareMathOperator{\de}{\mathrm{d}}
\DeclareMathOperator{\up}{\uparrow}
\DeclareMathOperator{\dn}{\downarrow}

%%%%%%%%%%%%%%%%%%%%%%%%%%%%%%%%%%%%%%%%%%%%%%%%%%%%%%%%%%%%%%%
% Derivadas
\newcommand{\dpf}[2]{\frac{d#1}{d#2}}
\newcommand{\dsf}[2]{\frac{d^2#1}{d#2^2}}
\newcommand{\opt}[1]{{#1}^{*}}

%%%%%%%%%%%%%%%%%%%%%%%%%%%%%%%%%%%%%%%%%%%%%%%%%%%%%%%%%%%%%%%
% Derivadas Parciais
\newcommand{\pdf}[2]{\frac{\partial#1}{\partial#2}}
\newcommand{\psdf}[2]{\frac{\partial^2#1}{\partial#2^2}}
\newcommand{\pcdf}[3]{\frac{\partial^2#1}{\partial#2 \partial#3}}

%%%%%%%%%%%%%%%%%%%%%%%%%%%%%%%%%%%%%%%%%%%%%%%%%%%%%%%%%%%%%%%
% 
\newcommand{\bm}[1]{\boldsymbol{#1}}
\newcommand{\mbs}[1]{\boldsymbol{#1}}
\newcommand{\mbf}[1]{\mathbf{#1}}

%%%%%%%%%%%%%%%%%%%%%%%%%%%%%%%%%%%%%%%%%%%%%%%%%%%%%%
